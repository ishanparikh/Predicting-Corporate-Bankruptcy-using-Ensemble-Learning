\label{chap:Conclusion}

% It is a very important  to accurately predict business failure in financial decision-making.
Bankruptcy prediction has long been regarded as a critical topic and has been studied extensively in the accounting and finance literature and now more recently in data science and artificial intelligence communities.

To the best of our knowledge, this paper puts forth a novel contribution to the bankruptcy prediction problem by using a ten-year rolling window model to predict bankruptcy on a one-year horizon. The Asian economy is often ignored in financial research - the paucity of available resources is representative of this. As Japan has the third-highest market capitalisation in the world, we chose to carry out our research on the Japanese economy.
After conducting a meticulous data analysis on the financial health of 4000 listed Japanese companies from 2000 to 2018, we compare three imputation techniques along with oversampling the minority class and then use ensemble learning to produce results that are in line with similar research on different economies \cite{barboza2017machine,le2018cluster}.

The biggest we overcame was aligning our data to one timeline. Since the companies being evaluated for bankruptcy don’t all follow GAAP standards, it was difficult to gather data (from nine separate files, each reporting a different category of financial data) and reorganise records such that a particular indicator was consistently reported for the same month/year.
The features on which we ultimately modelled bankruptcy prediction were not the indicators mentioned directly in our data, but are financial ratios that have been chosen based on their relevance and their frequency presented in seminal research.
They were thoroughly studied and validated and then generated by applying an arithmetical operation on a selection of two or more existing features. We have successfully documented our findings and put forth the best bankruptcy prediction model.


Research from \cite{paleologo2010subagging,brown2012experimental} suggests that using a single classifier cannot solve all problems effectively whereas ensemble models have been revealed to be promising in many credit risk and bankruptcy forecasting studies. 
Taking this into account, we perform meticulous experiments with ensemble classifiers like random forests, bagging and XGBoost. 

A leading contribution of our research is a detailed comparison between imputation techniques to estimate financial data. We submit a case to use Multiple Imputation by Chained Equations as for data estimation as it outperformed all other imputation methods by producing the optimal model performance for every machine learning model used in this study. We also propose the use of SMOTE as an over-sampling technique to increase the number of minority class (i.e., bankrupt) instances. Incorporating SMOTE increased in AUC of over 12\% and improves the F1-Score by over 18\%.

We began our research by establishing a valid baseline using Logistic Regression and then proceeded to use tree-based machine learning and ensemble models of increasing complexity. We started with Decision Trees and then explored ensemble methods like Random Forests, Bagging and XGBoost. 

Our approach has resulted in a considerable improvement in the quality of prediction performance. We conclude by reporting that the best model has an AUC score of 0.856 and an F1-Score of 0.798, on using MICE as the imputation technique and XGBoost as the learning model.
We also report that the adjusted share of equity in the financing assets, the current ratio and the liabilities turnover ratio are the most influential financial ratios in predicting bankruptcies of listed Japanese companies.

The presented model is designed to not be limited to listed Japanese companies but to represent a general framework that can be applied to any economic and financial dataset.


% \section{Future Work}
As bankruptcy prediction is a task that encapsulates such a vast domain, we would like to focus our attention towards researching the applicability of neural networks in bankruptcy prediction.
In particular, we would like to explore the potential of ensembles models that use neural networks as base learners, to test whether they can achieve higher performance as compared to the current tree-based algorithms we used in our research. Results on listed firms in Korea \cite{kim2010ensemble} indicates that the bagged and the boosted neural networks show the improved performance over traditional neural networks. 
Additionally, the use of shallow convolutional neural nets presented in Dr. Tiejun Ma's paper \cite{orimoloye2020comparing} have achieved high accuracy on predicting stock prices and we would like to see whether that hypothesis could be extended to predicting bankruptcy.



